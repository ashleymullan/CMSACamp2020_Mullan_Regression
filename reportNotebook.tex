\PassOptionsToPackage{unicode=true}{hyperref} % options for packages loaded elsewhere
\PassOptionsToPackage{hyphens}{url}
%
\documentclass[
]{article}
\usepackage{lmodern}
\usepackage{amssymb,amsmath}
\usepackage{ifxetex,ifluatex}
\ifnum 0\ifxetex 1\fi\ifluatex 1\fi=0 % if pdftex
  \usepackage[T1]{fontenc}
  \usepackage[utf8]{inputenc}
  \usepackage{textcomp} % provides euro and other symbols
\else % if luatex or xelatex
  \usepackage{unicode-math}
  \defaultfontfeatures{Scale=MatchLowercase}
  \defaultfontfeatures[\rmfamily]{Ligatures=TeX,Scale=1}
\fi
% use upquote if available, for straight quotes in verbatim environments
\IfFileExists{upquote.sty}{\usepackage{upquote}}{}
\IfFileExists{microtype.sty}{% use microtype if available
  \usepackage[]{microtype}
  \UseMicrotypeSet[protrusion]{basicmath} % disable protrusion for tt fonts
}{}
\makeatletter
\@ifundefined{KOMAClassName}{% if non-KOMA class
  \IfFileExists{parskip.sty}{%
    \usepackage{parskip}
  }{% else
    \setlength{\parindent}{0pt}
    \setlength{\parskip}{6pt plus 2pt minus 1pt}}
}{% if KOMA class
  \KOMAoptions{parskip=half}}
\makeatother
\usepackage{xcolor}
\IfFileExists{xurl.sty}{\usepackage{xurl}}{} % add URL line breaks if available
\IfFileExists{bookmark.sty}{\usepackage{bookmark}}{\usepackage{hyperref}}
\hypersetup{
  pdftitle={Report Notebook},
  pdfborder={0 0 0},
  breaklinks=true}
\urlstyle{same}  % don't use monospace font for urls
\usepackage[margin=1in]{geometry}
\usepackage{longtable,booktabs}
% Allow footnotes in longtable head/foot
\IfFileExists{footnotehyper.sty}{\usepackage{footnotehyper}}{\usepackage{footnote}}
\makesavenoteenv{longtable}
\usepackage{graphicx,grffile}
\makeatletter
\def\maxwidth{\ifdim\Gin@nat@width>\linewidth\linewidth\else\Gin@nat@width\fi}
\def\maxheight{\ifdim\Gin@nat@height>\textheight\textheight\else\Gin@nat@height\fi}
\makeatother
% Scale images if necessary, so that they will not overflow the page
% margins by default, and it is still possible to overwrite the defaults
% using explicit options in \includegraphics[width, height, ...]{}
\setkeys{Gin}{width=\maxwidth,height=\maxheight,keepaspectratio}
\setlength{\emergencystretch}{3em}  % prevent overfull lines
\providecommand{\tightlist}{%
  \setlength{\itemsep}{0pt}\setlength{\parskip}{0pt}}
\setcounter{secnumdepth}{-2}
% Redefines (sub)paragraphs to behave more like sections
\ifx\paragraph\undefined\else
  \let\oldparagraph\paragraph
  \renewcommand{\paragraph}[1]{\oldparagraph{#1}\mbox{}}
\fi
\ifx\subparagraph\undefined\else
  \let\oldsubparagraph\subparagraph
  \renewcommand{\subparagraph}[1]{\oldsubparagraph{#1}\mbox{}}
\fi

% set default figure placement to htbp
\makeatletter
\def\fps@figure{htbp}
\makeatother


\title{Report Notebook}
\author{}
\date{\vspace{-2.5em}July 1, 2020}

\begin{document}
\maketitle

\begin{verbatim}
## Loading required package: ggplot2
\end{verbatim}

\begin{verbatim}
## Registered S3 method overwritten by 'GGally':
##   method from   
##   +.gg   ggplot2
\end{verbatim}

\begin{verbatim}
## -- Attaching packages ---------------------------------------------------------- tidyverse 1.3.0 --
\end{verbatim}

\begin{verbatim}
## v tibble  3.0.1     v dplyr   1.0.0
## v tidyr   1.1.0     v stringr 1.4.0
## v readr   1.3.1     v forcats 0.5.0
## v purrr   0.3.4
\end{verbatim}

\begin{verbatim}
## -- Conflicts ------------------------------------------------------------- tidyverse_conflicts() --
## x dplyr::filter() masks stats::filter()
## x dplyr::lag()    masks stats::lag()
\end{verbatim}

\begin{verbatim}
## Loading required package: Matrix
\end{verbatim}

\begin{verbatim}
## 
## Attaching package: 'Matrix'
\end{verbatim}

\begin{verbatim}
## The following objects are masked from 'package:tidyr':
## 
##     expand, pack, unpack
\end{verbatim}

\begin{verbatim}
## Loaded glmnet 4.0-2
\end{verbatim}

\begin{verbatim}
## Warning: package 'coefplot' was built under R version 4.0.2
\end{verbatim}

\hypertarget{key-takeaways}{%
\section{Key Takeaways}\label{key-takeaways}}

Our analysis aimed to generate and interpret an efficient model of the
summary season statistics for 36 NBA teams ranging from the 2003-2004
season to 2018-2019. There were two kinds of variables, play frequencies
and success percentages. We modified the play frequencies to account for
seasons with shorter game play by scaling our variables down to an
individual game level. After extensive exploratory data analysis, we
discovered that the season statistics indicated a close game score on
average. This led us to try the average points per game metric as the
response variable in our initial models. Our final model included all of
the play frequency variables scaled to the average game level. We
discovered that less direct metrics, such as assists and rebounds,
strongly influence our model predictions. Future research would explore
the possibility of including more direct variables in our model and the
implications of our coefficients.

\includegraphics{reportNotebook_files/figure-latex/unnamed-chunk-3-1.pdf}

\hypertarget{introduction}{%
\section{Introduction}\label{introduction}}

\hypertarget{necessary-code-components-and-data-description}{%
\subsubsection{Necessary Code Components and Data
Description}\label{necessary-code-components-and-data-description}}

To generate this model, we used 7 R packages. These included
\emph{GGally}, \emph{ggcorrplot}, \emph{tidyverse}, \emph{glmnet},
\emph{broom}, \emph{coefplot}, and \emph{nbastatR}. The National
Basketball Association (NBA) released our data set of 479 observations
via the NBA stats website. It is a comprehensive summary of regular
season performance across 36 NBA teams ranging from the 2003-04 season
to the 2018-19 season. Originally, it included 28 variables consisting
of stats such as free throw percentage, win percentage, point
accumulation, attempted field goals, and other in-game stats. As we
progressed, we mutated 14 more variables to represent statistics on a
game level rather than across the season to improve interpretability.
\#\#\# Explanation of Variables The list of variables included in our
consideration may be found in the table below. The asterisk indicates
our response variable.

\begin{longtable}[]{@{}ll@{}}
\toprule
\begin{minipage}[b]{0.37\columnwidth}\raggedright
Variable Name\strut
\end{minipage} & \begin{minipage}[b]{0.57\columnwidth}\raggedright
Variable Description\strut
\end{minipage}\tabularnewline
\midrule
\endhead
\begin{minipage}[t]{0.37\columnwidth}\raggedright
avg\_score\_differential\strut
\end{minipage} & \begin{minipage}[t]{0.57\columnwidth}\raggedright
The accumulated score difference across the season divided by the number
of games in the season\strut
\end{minipage}\tabularnewline
\begin{minipage}[t]{0.37\columnwidth}\raggedright
avg\_assists\strut
\end{minipage} & \begin{minipage}[t]{0.57\columnwidth}\raggedright
The accumulated assists across the season divided by the number of games
in the season.\strut
\end{minipage}\tabularnewline
\begin{minipage}[t]{0.37\columnwidth}\raggedright
avg\_blocks\strut
\end{minipage} & \begin{minipage}[t]{0.57\columnwidth}\raggedright
The accumulated blocks across the season divided by the number of games
in the season\strut
\end{minipage}\tabularnewline
\begin{minipage}[t]{0.37\columnwidth}\raggedright
avg\_d\_rebounds\strut
\end{minipage} & \begin{minipage}[t]{0.57\columnwidth}\raggedright
The accumulated defensive rebounds across the season divided by the
number of games in the season\strut
\end{minipage}\tabularnewline
\begin{minipage}[t]{0.37\columnwidth}\raggedright
avg\_fouls\strut
\end{minipage} & \begin{minipage}[t]{0.57\columnwidth}\raggedright
The accumulated fouls across the season divided by the number of games
in the season\strut
\end{minipage}\tabularnewline
\begin{minipage}[t]{0.37\columnwidth}\raggedright
avg\_o\_rebounds\strut
\end{minipage} & \begin{minipage}[t]{0.57\columnwidth}\raggedright
The accumulated offensive rebounds across the season divided by the
number of games in the season\strut
\end{minipage}\tabularnewline
\begin{minipage}[t]{0.37\columnwidth}\raggedright
avg\_points*\strut
\end{minipage} & \begin{minipage}[t]{0.57\columnwidth}\raggedright
The accumulation of points across the season divided by the number of
games in the season\strut
\end{minipage}\tabularnewline
\begin{minipage}[t]{0.37\columnwidth}\raggedright
avg\_rebounds\strut
\end{minipage} & \begin{minipage}[t]{0.57\columnwidth}\raggedright
The accumulated total rebounds across the season divided by the number
of games in the season\strut
\end{minipage}\tabularnewline
\begin{minipage}[t]{0.37\columnwidth}\raggedright
avg\_steals\strut
\end{minipage} & \begin{minipage}[t]{0.57\columnwidth}\raggedright
The accumulated steals across the season divided by the number of games
in the season\strut
\end{minipage}\tabularnewline
\begin{minipage}[t]{0.37\columnwidth}\raggedright
avg\_turnovers\strut
\end{minipage} & \begin{minipage}[t]{0.57\columnwidth}\raggedright
The accumulated turnovers across the season divided by the number of
games in the season\strut
\end{minipage}\tabularnewline
\bottomrule
\end{longtable}

\hypertarget{exploratory-data-analysis}{%
\section{Exploratory Data Analysis}\label{exploratory-data-analysis}}

The first step in our research was to identify candidate response
variables. We decided upon average score differential and average points
per game. Next, we viewed the distribution of our intended response
variables. The histograms resulted in the distributions being normal
enough for us to feel confident in using either of them as our final
response variable. We checked the distributions of our other variables
as well which produced relatively normal results. After discovering
multiple variable distributions with somewhat concerning tail width, we
followed up with outlier testing. An outlier would have been classified
as a point outside of the bounding created by extending the benchmarks
of the first and third quartiles by the quantity of one and a half times
the interquartile range towards the extremes in each direction. We
created a scatter plot to arrive at a better understanding of the types
of games within the data set through the lens of average points per game
and average score differential. Three categories of games appeared:
those with a drastic win or loss, an average win or loss, and close
games. A game that was won or lost by 9 or more points on average was
considered a Big Loss/Win, between 3 and 9 points on average was
considered a Regular Loss/Win, and a difference of under 3 points was a
Close Game.

\includegraphics{reportNotebook_files/figure-latex/unnamed-chunk-4-1.pdf}

\hypertarget{modeling}{%
\section{Modeling}\label{modeling}}

When selecting the subset of variables for our final model, we
ultimately chose the average points per game value over the average
score differential as our response variable. The number of points in a
game is easily interpretable for coaches, and the rules of basketball
dictate that success requires a team to maximize the number of points
they score. Our beginning assumption was that variables measuring
percentage of successfully scoring plays such as field goal percentage
and free throw percentage would be positively correlated with average
points per game. We used correlation matrices to confirm our assumption
and found very high linear correlation coefficients between those
statistics. To uncover more interesting trends, we considered less
direct variables such as the counts for assists, fouls, rebounds,
steals, blocks, and turnovers. We eliminated candidate models with 16
fold cross validation using the leave one season out method. Our first
candidate model was the null model, which had a root mean squared error
of 6.070041. We next considered models with solely the offensive and
solely the defensive metrics, which had a root mean squared error of
4.842079 and 4.309872 respectively. Our best model incorporated both
offensive and defensive metrics. This model had a root mean squared
error of 4.189143. On the scale of average points per game, this is
relatively small. Our median residual was 0.0477. The four plots below
check the assumptions of our linear modeling. There is no clear pattern
in the residuals, and the Q-Q plot lies fairly close to the 45 degree
angle line. The leverage plot only indicates one potential outlier, and
previous checks indicate that the observation falls in the acceptable
range of values as determined by setting bounds via the interquartile
range. Our adjusted \(R^2\) value was 0.5397, which indicated a moderate
positive linear relationship between our predictor variables and the
average points per game value.

\includegraphics{reportNotebook_files/figure-latex/unnamed-chunk-5-1.pdf}
\includegraphics{reportNotebook_files/figure-latex/unnamed-chunk-5-2.pdf}
\includegraphics{reportNotebook_files/figure-latex/unnamed-chunk-5-3.pdf}
\includegraphics{reportNotebook_files/figure-latex/unnamed-chunk-5-4.pdf}

\hypertarget{results}{%
\section{Results}\label{results}}

All of the variables used in our model are on a frequency scale, so we
can consider the magnitude of the coefficients relative to each other
when interpreting our results. The coefficients of our linear model
appear in the table below.

\begin{longtable}[]{@{}ll@{}}
\toprule
Variable & Coefficient\tabularnewline
\midrule
\endhead
Intercept & 19.98\tabularnewline
Average Turnovers & -0.73\tabularnewline
Average Fouls & 0.38\tabularnewline
Average Assists & 1.31\tabularnewline
Average Rebounds & 1.28\tabularnewline
Average Steals & 1.08\tabularnewline
Average Blocks & -0.81\tabularnewline
Average Blocks Against & -0.87\tabularnewline
\bottomrule
\end{longtable}

Turnovers, blocks, and blocks have a negative relationship with the
number of points per game. This negative correlation may be attributed
to the fact that an instance of a turnover, block, or block against a
player would indicate that the team failed to score a basket on a
possession or shot attempt. In our model, assists and rebounds increase
points per game the most drastically, with steals not far behind. By
definition, an assist automatically precedes a basket. Thus, the mere
definition of the play supports its prominence as a key predictor of the
average points scored in a game. Defensive rebounds and steals force a
switch in possession, which allows the team an attempt to shoot and
score more points. Offensive rebounds, in contrast, allow for another
shot attempt on a current offensive possession. Fouls have the least
effect on our model, because it is uncertain that they will lead to a
basket. If they do, it is only worth 1 point.

\hypertarget{conclusion}{%
\section{Conclusion}\label{conclusion}}

Our final model helped us develop three important conclusions. We
observed that all of the frequency metrics in our data set contribute to
the accuracy of point predictions when scaled down to the individual
game level. In order to produce the most consistent model, we rejected
models that took solely offensive, defensive, or percentage statistics
into account. This approach aligns with the tactic of combining
offensive and defensive strategies in gameplay. We also recognized that
collinearity is a prominent hazard in basketball data. Collinearity
concerns arose when looking at field goal percentage, two point
percentage and three point percentage and again with offensive rebounds,
defensive rebounds, and total rebounds. Finally, we noted that close
games are fairly common in the NBA. Extensive exploratory data analysis
supported this assertion and motivated the choice of average points per
game as a response variable. The perspective of average points per game
over the course of a season benefits players and coaches, as the insight
that one play could be the difference between a win and a loss can shape
gameplay. In the future, we hope to consider new models that include
more direct game statistics such as field goal percentage, free throw
percentage, and win percentage. We would also like to further our
understanding of the implications of some of our coefficients on the
current model, such as the negative value for average blocks against
players on the team.

\end{document}
